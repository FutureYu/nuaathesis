\documentclass[
  lang=cn,
  degree=master,
  % zhuanshuo,  % 专硕打开
  % blindtrail, % 盲审打开
  % signname,   % 承诺书签字打开
  % signdate,   % 承诺书签日期打开
  blankleft     % 空白页没页眉页脚
  openright,twoside % 章节在奇数页,页码靠外侧
]{nuaathesis}

% Windows下可以加入以下:字体设置为宋体
\setCJKmainfont[AutoFakeBold=true]{SimSun} 
\renewcommand\songti{\CJKfamily{\CJKrmdefault}}

\raggedbottom
\graphicspath{{./fig/},{./logo/},{../logo/}}

\iffalse
  % 本块代码被上方的 iffalse 注释掉,如需使用,请改为 iftrue
  % 使用 Noto 字体替换中文宋体、黑体
  \setCJKfamilyfont{\CJKrmdefault}[BoldFont=Noto Serif CJK SC Bold]{Noto Serif CJK SC}
  \renewcommand\songti{\CJKfamily{\CJKrmdefault}}
  \setCJKfamilyfont{\CJKsfdefault}[BoldFont=Noto Sans CJK SC Bold]{Noto Sans CJK SC Medium}
  \renewcommand\heiti{\CJKfamily{\CJKsfdefault}}
\fi

\iffalse
  % 本块代码被上方的 iffalse 注释掉,如需使用,请改为 iftrue
  % 在 XeLaTeX + ctexbook 环境下使用 Noto 日文字体
  \setCJKfamilyfont{mc}[BoldFont=Noto Serif CJK JP Bold]{Noto Serif CJK JP}
  \newcommand\mcfamily{\CJKfamily{mc}}
  \setCJKfamilyfont{gt}[BoldFont=Noto Sans CJK JP Bold]{Noto Sans CJK JP}
  \newcommand\gtfamily{\CJKfamily{gt}}
\fi


% 设置基本文档信息,\linebreak 前面不要有空格,否则在无需换行的场合,中文之间的空格无法消除
\nuaaset{
  thesisid = {\secretize{1028716} 24-S000},   % 论文编号
  title = {\nuaathesis{} 快速上手\linebreak 示例文档},
  author = {nuaatug},
  college = {\TeX{} 学院},
  advisers = {Donald Knuth\quad 大师, tex.se 大牛们},
  % applydate = {二〇一八年六月}  % 默认当前日期
  %
  % 本科
  major = {\LaTeX{} 科学与技术},
  studentid = {131810299},
  classid = {应用技术},           % 班级的名称
  industrialadvisers = {Jack Ma}, % 企业导师,若无请删除或注释本行
  % 硕/博士
  majorsubject = {\LaTeX},
  researchfield = {\LaTeX 排版},
  libraryclassid = {TP371},       % 中图分类号
  subjectclassid = {080605},      % 学科分类号
}
\nuaasetEn{
  title = {\nuaathesis{} Quick Start\linebreak and Document Snippets},
  author = {nuaatug},
  college = {College of \TeX},
  majorsubject = {\LaTeX{} Typesetting},
  advisers = {Prof.~Donald Knuth, tex.se users},
  degreefull = {Master of Art and Engineering},
  % applydate = {June, 8012}
}

% 摘要
\begin{abstract}
    本文介绍如何使用\nuaathesis{} 文档类撰写南京航空航天大学学位论文。
    
    首先介绍如何获取并编译本文档,然后展示论文部件的实例,最后列举部分常用宏包的使用方法,取得的研究成果如下:
    
    \begin{enumerate}[leftmargin=0pt,labelsep=0pt,itemindent=2\ccwd+3.5ex,listparindent=2\ccwd,label={(\arabic*)\ }]
        \item 提出了一种 \LaTeX{} 写作方法。
        
        \item 提出了另一种 \LaTeX{} 写作方法。
    \end{enumerate}
\end{abstract}
\keywords{学位论文, 模板, \nuaathesis}

\begin{abstractEn}
    This document introduces \nuaathesis, the \LaTeX{} document class for NUAA Thesis.

    First, we show how to get the source code and compile this document.
    Then we provide snippets for figures, tables, equations, etc.
    Finally we enforce some usage patterns.
    The research results are as follows:

    \begin{enumerate}[leftmargin=0pt,itemindent=2.4em,labelsep=*,label=(\arabic*)]
        \item This thesis proposes a \LaTeX{} writing method.
        
        \item This thesis proposes another \LaTeX{} writing method.
    \end{enumerate}
\end{abstractEn}
\keywordsEn{NUAA thesis, document class, space is accepted here}

% 请按自己的论文排版需求,随意修改以下全局设置

\usepackage{subfig}
\usepackage{rotating}
\usepackage[usenames,dvipsnames]{xcolor}
\usepackage{tikz}
\usepackage{pgfplots}
\pgfplotsset{compat=1.16}
\pgfplotsset{
  table/search path={./fig/},
}
\usepackage{ifthen}
\usepackage{longtable}
\usepackage{siunitx}
\usepackage{listings}
\usepackage{multirow}
\usepackage{pifont}
\usepackage{fdsymbol}
\usepackage[linesnumbered,ruled,boxed,algochapter,noline,longend]{algorithm2e}
\usepackage{tabularx}
\usepackage{gensymb}
\usepackage{booktabs}
\usepackage{enumitem,calc}

\lstdefinestyle{lstStyleBase}{%
  basicstyle=\small\ttfamily,
  aboveskip=\medskipamount,
  belowskip=\medskipamount,
  lineskip=0pt,
  boxpos=c,
  showlines=false,
  extendedchars=true,
  upquote=true,
  tabsize=2,
  showtabs=false,
  showspaces=false,
  showstringspaces=false,
  numbers=left,
  numberstyle=\footnotesize,
  linewidth=\linewidth,
  xleftmargin=\parindent,
  xrightmargin=0pt,
  resetmargins=false,
  breaklines=true,
  breakatwhitespace=false,
  breakindent=0pt,
  breakautoindent=true,
  columns=flexible,
  keepspaces=true,
  framesep=3pt,
  rulesep=2pt,
  framerule=1pt,
  backgroundcolor=\color{gray!5},
  stringstyle=\color{green!40!black!100},
  keywordstyle=\bfseries\color{blue!50!black},
  commentstyle=\slshape\color{black!60}}

%\usetikzlibrary{external}
%\tikzexternalize % activate!

\newcommand\cs[1]{\texttt{\textbackslash#1}}
\newcommand\pkg[1]{\texttt{#1}\textsuperscript{PKG}}
\newcommand\env[1]{\texttt{#1}}

\DeclareSubrefFormat{myparens}{#1(#2)}
\captionsetup[subfloat]{subrefformat=myparens}

\renewcommand{\algorithmcfname}{算法} 
\SetKwInOut{Input}{输入}
\SetKwInOut{Output}{输出}

\theoremstyle{nuaaplain}
\nuaatheoremchapu{definition}{定义}
\nuaatheoremchapu{assumption}{假设}
\nuaatheoremchap{exercise}{练习}
\nuaatheoremchap{nonsense}{胡诌}
\nuaatheoremchap{theorem}{定理}
\nuaatheoremg[句]{lines}{句子}


\begin{document}

\makecover
\makedeclare
\frontmatter
\makeabstract
% 如果需要调整目录层级数量的话,取消下一行注释,数字含义: 0=chapter, 1=section, 2=subsection
% \setcounter{tocdepth}{1}
\expandafter\nuaatableofcontents
\expandafter\nuaalistoffigurestables
% 注释表和缩略词,硕博论文用。
% 《要求》没有规定内容格式,按照自己的喜好来改吧。
% 注意,表格里的文字不要太长哦。

\chapter*{注释表}

\noindent\begin{tabu} to \textwidth {|X[l]|p{4.5cm}|X[l]|p{4.5cm}|}\hline
$A, A_0$ & 状态方程矩阵 & $e$ & 误差绝对值 \\ \hline
$a$ & 重心到前轴的距离 & $e_i$ & 误差变化率 \\ \hline
$a_0, a_1, a_2, a_3$ & 多项式系数 & $F(\omega)$ & 多项式 \\ \hline
$a_{c0}$ & 加速度变量 & $F_i, \theta _i$ & Fadeev递归算法中间变量 \\ \hline

\multirow{2}{*}{$a_{s1}, a_{s0}$} & \multirow{2}{4.5cm}{连轴器及传动轴简化模型传递系数} &
$F_X$ & 汽车总制动力 \\ \cline{3-4}
& & $F_Y$ & 汽车总侧向力 \\ \hline

$a_y$ & 横向加速度 & $f_b$ & 轮胎制动力 \\ \hline
$a_{yc}$ & 横向加速度极限值 & $f_{bi}, f_{ci}$ & 各轮制动力和侧偏力 \\ \hline
$\tilde{a}_0, \tilde{a}_1, \tilde{a}_2, \tilde{a}_3$ & 多项式系数 & $G$ & 状态方程矩阵 \\ \hline
$B, B_0, B_1$ & 状态方程矩阵 & $g$ & 重力加速度 \\ \hline
$B_{w1}, B_{w2}$ & 状态方程矩阵 & $H$ & 汽车重心高度 \\ \hline
$b$ & 重心到后轴的距离 & $H(j \omega)$ & 频响函数 \\ \hline
$b_0, b_1, b_2, b_3$ & 多项式系数 & $h$ & 汽车重心到侧倾中心的距离 \\ \hline
$b_m$ &电机阻尼比系数 & $h_r$ & 汽车侧倾中心高度 \\ \hline
\end{tabu}

\chapter*{缩略词}

\noindent\begin{tabu} to \textwidth {|X[1,c]|X[4,c]|}\hline
缩略词 & 英文全称 \\ \hline
WSN & Wireless Sensor Networks \\ \hline
CAM & Center Angle Method \\ \hline
LEACH & Low-Energy Adaptive Clustering Hierarchy \\ \hline
\end{tabu}


\mainmatter

% 自由修改添加的章
% 本文件是示例论文的一部分
% 论文的主文件位于上级目录的 `bachelor.tex` 或 `master.tex`

\chapter{快速上手}

\section{欢迎}

欢迎使用 \nuaathesis,本文档将介绍如何利用 \nuaathesis 模板进行学位论文写作,
我们假设读者有 \LaTeX 英文写作经验,并会使用搜索引擎解决常见问题。

本模板的源代码托管在 \url{https://github.com/nuaatug/nuaathesis},
欢迎来提 issue/PR。

\section{\LaTeX 环境准备}

由于本模板使用了大量宏包,因此对 \LaTeX 环境有不少要求。
推荐使用以下打 \ding{51} 的 \LaTeX 发行版:
\begin{itemize}
\item[\ding{51}]\TeX~Live 请安装以下 collection:langchinese, latexextra, science, pictures, fontsextra;\\
如果觉得安装体积太大的话,可以看 \texttt{.ci/texlive.pkgs} 列出的所需宏包;
\item[\ding{51}]MiK\TeX 请祈祷国内的镜像服务器不抽风;如果它抽风了,建议隔天再试; \\
因为 MiK\TeX{} 能自动下载安装宏包,非常推荐 Windows 用户使用。
\item[\ding{53}]CTeX (\url{http://www.ctex.org/}) 不推荐,可能宏包缺失、版本过旧导致无法编译。
\end{itemize}

\section{编译模板和文档}

只有在找不到 \verb|nuaathesis.cls| 文件的时候,才需要执行本步骤。

进入模板的根目录,运行 \verb|build.bat|(Windows) 或 \verb|build.sh|(其他系统),
它会生成模板 \verb|nuaathesis.cls| 以及对应的文档 \verb|nuaathesis.pdf|。

\section{使用模板}

论文写作时,请确认\textbf{论文的目录}下有以下文件:
\begin{itemize}
  \item \verb|nuaathesis.cls| 文档模板;
  \item \verb|nuaathesis.bst| 参考文献格式(如果使用 biber 来生成参考文献的话);
  \item \verb|logo/| 文件夹,内含一些图标;
\end{itemize}

如果论文目录下没有这些文件的话,请从本模板根目录复制一份。

\section{开始写作}

最方便的开始方法,莫过于修改现有的文稿。因此推荐直接修改本文档:
\begin{itemize}
  \item \verb|bachelor.tex| 或 \verb|master.tex| 主文件,定义了文档包含的内容。建议删除并只保留其中一个主文件;
  \item \verb|global.tex| 里面定义文档的信息,导入一些宏包,并设置全局使用的宏;
  \item \verb|content/| 文件夹,按章节拆分的文档内容,这里;
  \item \verb|ref/| 文件夹,内含参考文献数据库;
\end{itemize}

修改完成后,使用 \verb|latexmk -xelatex bachelor| 进行编译。
如果需要使用图形界面的编辑器的话,请继续阅读本节内容。

\subsection{使用 TeXstudio}
\begin{enumerate}
\item 打开主文件 \verb|bachelor.tex| 或 \verb|master.tex|;
\item 菜单 Options > Configure TeXstudio 对话框;
\item 左侧选择 Build,右侧将 Default Compiler 修改为 Latexmk;
\item 确认即可。
\end{enumerate}

\subsection{使用 vscode}
\begin{enumerate}
\item 打开论文目录;
\item 安装 LaTeX Workshop 插件;
\item 打开论文的主文件 \verb|bachelor.tex| 或 \verb|master.tex|,删除没有用到的主文件;
\item 使用 LaTeX Workshop 插件提供的编译命令编译文档。
\end{enumerate}

\section{打印论文}

如果论文需要双面打印的话,请务必修改文档类选项,编译双面打印用的 PDF 文件。
具体地说,在主文件的头部,将 \texttt{openany, oneside} 改成 \texttt{openany, twoside}。

\chapter{特色功能}

本章节介绍由 \nuaathesis{} 提供的特有的宏。

\section{定理环境}

\nuaathesis{} 没有定义任何定理环境,
但提供了三个宏 \cs{nuaatheorem(g|chap|chapu)} 来定义不同编号方法的定理环境。
\begin{enumerate}
  \item \cs{nuaatheoremg} 的编号只有一个数字;
  \item \cs{nuaatheoremchap} 的编号由“章节.序号”构成,不同的定理环境的编号是独立的,
  它们的数字编号会重复,如“\autoref{ex:oneplus}”后面可能出现“\autoref{non:dora}”;
  \item \cs{nuaatheoremchapu} 的编号也是由“章节.序号”构成,
  但它们的数字编号是统一的,同一个数字不会重复出现(仅限用\cs{nuaatheoremchapu}声明的定理环境之间)。
  如“\autoref{def:distance}”后面\textbf{不会}出现“假设~2.1”,但可能出现“定义~2.2”或“\autoref{assume:fail}”;
\end{enumerate}

由于学校没有规定计数的编号,所以所有的定理环境应该由作者来决定编号方式,
这也意味着所有的定理环境都要由作者来定义(这不是 \nuaathesis{} 在偷懒哦)。

顺便一提,在同一章里同时出现两种编号方式的定理环境,很可能造成混乱,
所以请合理安排定理环境的编号方式。以下开始举栗子。

\subsection*{样例}

\begin{definition}[欧几里得距离]
\label{def:distance}
点$\mathbf{p}$与点$\mathbf{q}$的\textbf{欧几里得距离},是连接该两点的线段($\overline{\mathbf{pq}}$)的长度。

在笛卡尔坐标系下,如果 $n$维欧几里得空间下的两个点 $\mathbf{p}=(p_1, p_2, \dots, p_n)$ 与点
$\mathbf{q} = (q_1, q_2, q_3, \dots, q_n)$,那么点$\mathbf{p}$与点$\mathbf{q}$的距离,
或者点$\mathbf{q}$与点$\mathbf{p}$的距离,由以下公式定义:
\begin{align}
\label{equ:1}
d(\mathbf{p},\mathbf{q}) = d(\mathbf{q},\mathbf{p}) & = \sqrt{(q_1-p_1)^2 + (q_2-p_2)^2 + \cdots + (q_n-p_n)^2} \\
\label{equ:2}
& = \sqrt{\sum_{i=1}^n (q_i-p_i)^2}
\end{align}
\end{definition}

\begin{proof}
由\cs{nuaatheorem(g|chap|chapu)}定义的定理环境支持 \cs{autoref},
比如在\autoref{def:distance}中,\autoref{equ:2}是\autoref{equ:1}的简写。

但是 \cs{autoref} 只能在 \cs{ref} 加上前缀,无法加上后缀。
所以上一句话的后半部分,更推荐手工来写标注 “(\ref{equ:2}) 是 (\ref{equ:1}) 的简写”。

定理环境里面可以换行,不过证明与其他定理环境稍有不同,它是单独定义实现的,
因此末尾会有一个(帅气的) QED 符号。
\end{proof}

\begin{assumption}
\label{assume:fail}
假设本身就不成立
\end{assumption}

\begin{lines}
\label{s1}
例句1
\end{lines}

\section{参考文献}
\label{sec:bib}
参考文献应该以上标的形式标注于论述之后,就像这样:

\begin{itemize}
\item 研究表明\cite{r1},早睡早起有益身体健康。
\item 如果想同时引用多个文献\cite{r2,r3,r4,r6},只需要在 \verb|cite{}| 中用逗号分开\texttt{citeKey}就好。
\end{itemize}

本模板保留了 \cquthesis{} 里的 \texttt{inlinecite},但请注意它不符合学校的要求,无论本科还是硕士、博士,
请\textbf{谨慎}使用:
文献\inlinecite{r6}表明,文献\inlinecite{r7,r8,r9}所述的情况是有理论依据的。

\nuaathesis 格式测试,学校的参考文献格式并不是 GB7714-2015,所以追加一些测试样例。
《要求》里列出的格式有:
\begin{enumerate}
  \item 连续出版物\cite{n11,n12}:[序号]作者.文题.刊名,年,卷号(期号):起~止页码.
  \item 专译集\cite{n21,n22}:[序号]作者.书名(译者).出版地:出版者,出版年:起~止页码.
  \item 论文集\cite{n31,n32}:[序号]作者.文题.编者,文集名,出版地:出版者,出版年:起~止页码.
  \item 学位论文\cite{n41,n42,n43}:[序号]姓名.文题,[XX学位论文].授予单位所在地:授予单位,授予年.
  \item 专利\cite{n51,n52,n53}:[序号]申请者.专利名,国名,专利文献种类,专利号,出版日期.
  \item 技术标准\cite{n61,n62,n63}:[序号]发布单位,技术标准代号,技术标准名称,出版地:出版者,出版日期.
\end{enumerate}

注:目前实现的格式仍然与《要求》有点差异:
\begin{enumerate}
  \item 《要求》里论文集的编者、文集名、出版地是逗号分隔,而目前是点号分隔;
  \item 《要求》的学位论文用中文注明学位,目前没实现;
  \item 在信息缺失的情况下,《要求》貌似直接把对应字段省略,目前仍显示“XX不详”。
\end{enumerate}

\chapter{定理环境·下}

本章演示使用 \cs{nuaatheoremchap} 定义的定理环境,注意它们的数字编号是可以重复的。

\section{演示一级标题}
\subsection{演示二级标题}
\subsubsection{演示三级标题}

\begin{nonsense}
\label{non:dora}
哆啦A梦写的论文被拒稿的可能性很高
\footnote{出处:\url{https://www.math.kyoto-u.ac.jp/~arai/latex/presen2.pdf} 的最后一页}。
\end{nonsense}

\begin{exercise}
\label{ex:oneplus}
证明$1+1 = 2$。
\footnote{Testing footnote with English spaces}
\end{exercise}

\begin{nonsense}[右边的胡诌是真的]
“练习”与“胡诌”定理环境的编号是相互独立的,它们的数字编号允许重复,
如“\autoref{non:dora}”和“\autoref{ex:oneplus}”。
\end{nonsense}

\begin{exercise}
按照本文所演示的方法,利用 \cs{nuaatheorem(g|chap|chapu)} 来定义您的论文中所需要的定理环境。
\end{exercise}

\begin{lines}
\label{s2}
例句2
\end{lines}

\autoref{s2} 没有章节编号,它是全局编号的,它可以用在外国语学院论文中来枚举例句。

\chapter{使用示例}

本章介绍一些常用的宏包的常用方法,希望能为读者写作时提供参考。

\section{插图}

首先讨论一下插图的格式,在 \LaTeX{} 环境下,
\begin{enumerate}
\item 推荐使用宏包来绘制插图,如 \pkg{tikz},它兼容所有 \LaTeX{} 环境,
字体能与全文统一,质量最佳,但是需要的学习成本较大。
请务必先阅读 \pkg{tikz} 文档的第1章教程,
然后可以去 texample\footnote{\url{http://texample.net/tikz}} 等网站上找类似的例子,
也可以使用 GeoGebra\footnote{\url{https://www.geogebra.org}} 之类的工具来生成\TeX 代码,
效果可以参见\autoref{fig:tikzrot};
\item 其次推荐使用其他绘图工具生成的 \verb|PDF|、 \verb|EPS| 格式的矢量图,
\verb|svg| 格式可以通过 inkscape 软件转换成带 \TeX{}文本代码的 \verb|PDF|。效果可以参见\autoref{fig:logo};
\item 当然,\verb|PNG|、 \verb|jpeg| 之类的位图格式也能做插图;
\item 最后,不要忘记论文是\textbf{单色印刷}的,请确保插图在黑白打印的情况下的清晰度。
\end{enumerate}

\begin{figure}[!htb]
    \centering
    \includegraphics[width=0.6\textwidth]{nuaa-logo.pdf}
    \caption{一个校徽}
    \label{fig:logo}
\end{figure}

如果需要多个插图共用一个题注的话,需要加载额外的宏包,
一般选用 \pkg{subcaption} 或 \pkg{subfig},这两个宏包是互斥的。
需要注意的是 \pkg{subcaption} 貌似与 \pkg{geometry} 有些冲突,
会导致多行的图表的最后一行无法居中,而 \pkg{geometry} 是设置页边距的必用宏包。
所以个人推荐使用  \pkg{subfig},效果可以参考\autoref{fig:sub2}。

\begin{figure}[!htb]
  \subfloat[左边的大校徽\label{fig:sub1}]{\includegraphics[width=4cm]{nuaa-logo.pdf}}\quad
  \subfloat[短标题:小校徽][小校徽,题注很长,不过请各位放心,它会自动换行\label{fig:sub2}]
  {\includegraphics[width=3cm]{nuaa-logo.pdf}}
  \caption{包含两张图片的插图}
  \label{fig:subfigs}
\end{figure}


如果真的需要让十几张图片共用一个题注的话,
需要手工拆分成多个 \env{float} 并用 \cs{ContinuedFloat} 来拼接,
不过直接多次使用 \cs{caption} 会在图表清单里产生多个重复条目,需要一点点小技巧
(设置图表目录标题为空)。
建议将浮动位置指定为 \verb|t|,以确保分散至多页的图能占用整个页面,手工分页才能靠谱。
效果可以参见\autoref{fig:subfigss} 的\autoref{fig:logo6}。

\begin{figure}[t]
  \subfloat[校徽$\times 1$]{\includegraphics[width=4cm]{nuaa-logo.pdf}}\quad
  \subfloat[校徽$\times 2$]{\includegraphics[width=.4\textwidth]{nuaa-logo.pdf}}\\
  \subfloat[校徽$\times 3$]{\includegraphics[width=.4\textwidth]{nuaa-logo.pdf}}\quad
  \subfloat[校徽$\times 4$]{\includegraphics[width=4cm]{nuaa-logo.pdf}}
  \caption{包含多张图片的插图}
  \label{fig:subfigss}
\end{figure}
\begin{figure}[t]
  \ContinuedFloat
  \subfloat[校徽$\times 5$]{\includegraphics[width=4cm]{nuaa-logo.pdf}}\quad
  \subfloat[校徽$\times 6$ \label{fig:logo6}]{\includegraphics[width=4cm]{nuaa-logo.pdf}}\\
  \subfloat[校徽$\times 7$]{\includegraphics[width=4cm]{nuaa-logo.pdf}}\quad
  \subfloat[校徽$\times 8$]{\includegraphics[width=4cm]{nuaa-logo.pdf}}
  % 指定图表清单中的标题为[],即可将其消除,避免目录中出现重复条目
  \caption[]{包含多张图片的插图(续)}
\end{figure}

如果需要插入一张很大的图片的话,可以使用 \pkg{rotating} 提供的 \env{sidewaysfigure},
它能将插图放置在单独的页面上,如果文档使用 \verb|twoside| 选项的话,它会根据页面方向,
设置 \ang{90} 或 \ang{270} 旋转,可能需要编译两遍才能设置正确的旋转方向。
不过可能有一个问题,\env{sidewaysfigure} 中使用 \cs{subfloat} 可能无法准确标号,
需要手工重置 \texttt{subfigure} 计数器。
效果参见\autoref{fig:fullpage1} 和\autoref{fig:fullpage2}。

\setcounter{subfigure}{0}
\begin{sidewaysfigure}
  \subfloat[First caption\label{fig:fp1}]{\includegraphics[width=.8\textheight]{nuaa-jianqi.pdf}} \\
  \subfloat[Second caption]{\includegraphics[height=2cm]{nuaa-jianqi.pdf}}
  \caption{一幅占用完整页面的图片}
  \label{fig:fullpage1}
\end{sidewaysfigure}

\setcounter{subfigure}{0}
\begin{sidewaysfigure}
  \subfloat[First caption]{\includegraphics[height=2cm]{nuaa-jianqi.pdf}} \\
  \subfloat[Second caption]{\includegraphics[width=.8\textheight]{nuaa-jianqi.pdf}}
  \caption{又一幅占用完整页面的图片}
  \label{fig:fullpage2}
\end{sidewaysfigure}

\section{表格}

由于封面页,本模板预先加载了 \pkg{array} 和 \pkg{tabu},如果需要其他表格的宏包,
请自行加载。

如果需要插入一个简易的表格,可以只使用 \env{tabular} 环境,如\autoref{tab:city}。
\begin{table}[!htb]
  \caption[城市人口]{城市人口数量排名 (source: Wikipedia)\label{tab:city}}
  \begin{tabular}{lr}
    \toprule
    城市 & 人口 \\
    \midrule
    Mexico City & 20,116,842\\
    Shanghai & 19,210,000\\
    Peking & 15,796,450\\
    Istanbul & 14,160,467\\
    \bottomrule
  \end{tabular}
\end{table}

也可以使用 \env{tabu} 环境,它可以更灵活地设置列宽,但它有一些 bug,如\autoref{tab:tabu}。
\begin{table}[!htb]
  \caption{\env{tabu} 注意事项 \label{tab:tabu}}
  \begin{tabu} to .9\textwidth {XX[2]<{\strut}} \toprule
    默认列 & 有修正的列 \\ \midrule
    \env{tabu} 的 bug? \par This line is BAD & 注意左侧最后一行后的垂直空格 \\ \midrule
    注意对比最后一行 &
      bug 会影响多行的 \env{tabu} 表格 \par
      bug 的修正方法是在段落后面加 \cs{strut} \par
      This line is Good \\ \midrule
    垂直居中没效果 & 改用 \env{tabular} \\ \midrule
    与新版 \pkg{array} 不兼容 & 谨慎使用,切勿用 \texttt{tabu spread} \\ \bottomrule
  \end{tabu}
\end{table}

如果需要对某一列的小数点对齐,或者带有单位,或者需要做四舍五入的处理,可以尝试配合 \pkg{siunitx} 一起使用。
非常推荐看一下 \pkg{siunitx} 文档的,至少看一下“Hints for using siunitx”一节的输出结果,
\autoref{tab:xmpl:mixed} 来自于该文档的 9.11 节 Creating a column with numbers and units。

% \begin{table}[!htb]
%   \caption{Tables where numbers have different units}
%   \label{tab:xmpl:mixed}
%   \begin{tabular}
%     {
%       @{}
%       >{$}l<{$}
%       S[table-format = 3.3(1)]
%       S[table-format = 3.3(1)]
%       @{}
%     }
%     \toprule
%       & {One} & {Two} \\
%     \midrule
%     a / \unit{\pm}         &  123.4(2) &   567.8(4) \\
%     \beta / \unit{\degree} & 90.34(4)  & 104.45(5)  \\
%     \mu / \unit{\per\mm}   &  0.532    &   0.894    \\
%     \bottomrule
%   \end{tabular}
%   \hfil
%   \begin{tabular}
%     {
%       @{}
%       S[table-format=1.3]@{\,}
%       >{\collectcell\unit}l<{\endcollectcell}
%       @{}
%     }
%     \toprule
%     \multicolumn{2}{@{}c}{Heading} \\
%     \midrule
%     1.234 & \metre   \\
%     0.835 & \candela \\
%     4.23  & \joule\per\mole \\
%     \bottomrule
%   \end{tabular}
% \end{table}

如果表格内容很多,导致无法放在一页内的话,需要用 \env{longtable} 或 \env{longtabu} 进行分页。
\autoref{tab:performance} 是来自 \cquthesis{} 的一个长表格的例子。

\begin{longtable}[c]{c*{6}{r}}
	\caption[实验数据]{实验数据,这个题注十分的长,注意这在索引中的处理方式,还有 \cs{caption} 后面的双反斜杠}\label{tab:performance}\\
	\toprule
	\multirow{2}{*}{测试程序} & \multicolumn{1}{c}{正常运行} & \multicolumn{1}{c}{同步} & \multicolumn{1}{c}{检查点} & \multicolumn{1}{c}{卷回恢复}
	& \multicolumn{1}{c}{进程迁移} & \multicolumn{1}{c}{检查点} \\
	& \multicolumn{1}{c}{时间 (s)}& \multicolumn{1}{c}{时间 (s)}&
	\multicolumn{1}{c}{时间 (s)}& \multicolumn{1}{c}{时间 (s)}& \multicolumn{1}{c}{时间 (s)}& \multicolumn{1}{c}{文件 (KB)} \\ \midrule
	\endfirsthead
	\multicolumn{7}{c}{\nuaafontcaption 续表~\thetable\hskip1em 实验数据}\\
	\toprule
	\multirow{2}{*}{测试程序} & \multicolumn{1}{c}{正常运行} & \multicolumn{1}{c}{同步} & \multicolumn{1}{c}{检查点} & \multicolumn{1}{c}{卷回恢复}
	& \multicolumn{1}{c}{进程迁移} & \multicolumn{1}{c}{检查点} \\
	& \multicolumn{1}{c}{时间 (s)}& \multicolumn{1}{c}{时间 (s)}&
	\multicolumn{1}{c}{时间 (s)}& \multicolumn{1}{c}{时间 (s)}& \multicolumn{1}{c}{时间 (s)}& \multicolumn{1}{c}{文件(KB)} \\ \midrule
	\endhead
	\hline
	\multicolumn{7}{r}{续下页}
	\endfoot
	\endlastfoot
	CG.A.2 & 23.05 & 0.002 & 0.116 & 0.035 & 0.589 & 32491 \\
	CG.A.4 & 15.06 & 0.003 & 0.067 & 0.021 & 0.351 & 18211 \\
	CG.A.8 & 13.38 & 0.004 & 0.072 & 0.023 & 0.210 & 9890 \\
	CG.B.2 & 867.45 & 0.002 & 0.864 & 0.232 & 3.256 & 228562 \\
	CG.B.4 & 501.61 & 0.003 & 0.438 & 0.136 & 2.075 & 123862 \\
	CG.B.8 & 384.65 & 0.004 & 0.457 & 0.108 & 1.235 & 63777 \\
	MG.A.2 & 112.27 & 0.002 & 0.846 & 0.237 & 3.930 & 236473 \\
	MG.A.4 & 59.84 & 0.003 & 0.442 & 0.128 & 2.070 & 123875 \\
	MG.A.8 & 31.38 & 0.003 & 0.476 & 0.114 & 1.041 & 60627 \\
	MG.B.2 & 526.28 & 0.002 & 0.821 & 0.238 & 4.176 & 236635 \\
	MG.B.4 & 280.11 & 0.003 & 0.432 & 0.130 & 1.706 & 123793 \\
	MG.B.8 & 148.29 & 0.003 & 0.442 & 0.116 & 0.893 & 60600 \\
	LU.A.2 & 2116.54 & 0.002 & 0.110 & 0.030 & 0.532 & 28754 \\
	LU.A.4 & 1102.50 & 0.002 & 0.069 & 0.017 & 0.255 & 14915 \\
	LU.A.8 & 574.47 & 0.003 & 0.067 & 0.016 & 0.192 & 8655 \\
	LU.B.2 & 9712.87 & 0.002 & 0.357 & 0.104 & 1.734 & 101975 \\
	LU.B.4 & 4757.80 & 0.003 & 0.190 & 0.056 & 0.808 & 53522 \\
	LU.B.8 & 2444.05 & 0.004 & 0.222 & 0.057 & 0.548 & 30134 \\
	CG.B.2 & 867.45 & 0.002 & 0.864 & 0.232 & 3.256 & 228562 \\
	CG.B.4 & 501.61 & 0.003 & 0.438 & 0.136 & 2.075 & 123862 \\
	CG.B.8 & 384.65 & 0.004 & 0.457 & 0.108 & 1.235 & 63777 \\
	MG.A.2 & 112.27 & 0.002 & 0.846 & 0.237 & 3.930 & 236473 \\
	MG.A.4 & 59.84 & 0.003 & 0.442 & 0.128 & 2.070 & 123875 \\
	MG.A.8 & 31.38 & 0.003 & 0.476 & 0.114 & 1.041 & 60627 \\
	MG.B.2 & 526.28 & 0.002 & 0.821 & 0.238 & 4.176 & 236635 \\
	MG.B.4 & 280.11 & 0.003 & 0.432 & 0.130 & 1.706 & 123793 \\
	MG.B.8 & 148.29 & 0.003 & 0.442 & 0.116 & 0.893 & 60600 \\
	LU.A.2 & 2116.54 & 0.002 & 0.110 & 0.030 & 0.532 & 28754 \\
	LU.A.4 & 1102.50 & 0.002 & 0.069 & 0.017 & 0.255 & 14915 \\
	LU.A.8 & 574.47 & 0.003 & 0.067 & 0.016 & 0.192 & 8655 \\
	LU.B.2 & 9712.87 & 0.002 & 0.357 & 0.104 & 1.734 & 101975 \\
	LU.B.4 & 4757.80 & 0.003 & 0.190 & 0.056 & 0.808 & 53522 \\
	LU.B.8 & 2444.05 & 0.004 & 0.222 & 0.057 & 0.548 & 30134 \\
	EP.A.2 & 123.81 & 0.002 & 0.010 & 0.003 & 0.074 & 1834 \\
	EP.A.4 & 61.92 & 0.003 & 0.011 & 0.004 & 0.073 & 1743 \\
	EP.A.8 & 31.06 & 0.004 & 0.017 & 0.005 & 0.073 & 1661 \\
	EP.B.2 & 495.49 & 0.001 & 0.009 & 0.003 & 0.196 & 2011 \\
	EP.B.4 & 247.69 & 0.002 & 0.012 & 0.004 & 0.122 & 1663 \\
	EP.B.8 & 126.74 & 0.003 & 0.017 & 0.005 & 0.083 & 1656 \\
	\bottomrule
\end{longtable}

\section{数字与国际单位}

本模板预加载 \pkg{siunitx} 来格式化文中的内联数字,该宏包有大量可定制的参数,
请务必阅读其文档,并在文档导言部分设置格式。

\begin{itemize}
  \item 旋转角度为 \ang{90}、\ang{270}
  \item 分辨率 1920x1080 的像素数量约为 \num{2.07e6}
  \item 电脑显示器的像素间距为 \SI{1.8}{\nm}、\SI{180}{\um} 还是 \SI{18}{\mm}?
  \item 重力加速度 $g=\SI{9.8}{\kg\per\square\second}$、
  $g=\SI[inter-unit-product=\ensuremath{{}\cdot{}}]{9.8}{\kg\per\square\second}$,
  亦或 $g=\SI[per-mode=symbol]{9.8}{\kg\per\square\second}$
\end{itemize}

\section{中英文之间空格}

很遗憾,目前 \LaTeX{} 和 \CTeX{} 虽然能处理普通汉字与英文之间的间隔,
但是汉字与宏之间的空格仍然需要手工调整,请务必按以下的规则撰写原稿:
\begin{itemize}
  \item[\ding{51}] 如\autoref{fig:sub2} 所示:\verb|如\autoref{fig:sub2} 所示|,这个宏返回的是“图 x.xx”,
  所以前面两个汉字之间不能加空格,后面数字与汉字之间必须加空格;
  \item[\ding{51}] 距离为 1.7~个天文单位:\verb|距离为 1.7~个天文单位|,前面可以不加空格(\CTeX 会修正),
  后面必须加 \verb|~| 以防止在 “1.7”与“个”之间换行。此时更推荐写成 \SI{1.7}{au}:\verb|\SI{1.7}{au}|。
\end{itemize}

\section{算法}

\begin{algorithm}[!htb]
    \caption{计算混淆位置}
    \label{Alg:CalculateConfusedLocation}
    \Input{$\gamma$, $k$, $x_w$, $y_w$}
    \Output{$x_w^\prime$, $y_w^\prime$}
    初始化空集合 $X, Y$, $count=0$

    \While{$count < k$}  {
        \Repeat{$\sqrt{(x-x_w)^2+(y-y_w)^2} \le \gamma$}{
            随机生成 $x\in(x-\gamma,x+\gamma)$

            随机生成 $y\in(y-\gamma,y+\gamma)$
        }
        添加 $x$ 到 $X$, 添加 $y$ 到 $Y$, $count++$
    }
    $x_w^\prime={\rm sum}(X)/k$, $y_w^\prime={\rm sum}(Y)/k$

    \Return{$x_w^\prime$, $y_w^\prime$, $\gamma$} 
\end{algorithm}

\appendix
% 如果需要附录的话,在这里 include
% \include{content/ex_postscript}

\backmatter
% 如果参考文献使用 biber
\bibliographystyle{nuaabib}   % 参考文献的样式
\bibliography{bib/ref}   % 参考文献,即 bib/ref.bib 文件(纯文本)

\chapter[致谢]{致\hskip 2\ccwd{}谢}

\secretize{在此感谢对本论文作成有所帮助的人。}

\chapter{在学期间的研究成果及发表的学术论文}

% 不要在意为什么没有 section,因为只是套用一下格式

\subsection*{攻读硕士学位期间发表(录用)论文情况}

\begin{enumerate}
    \item \textbf{\secretize{AAA B}}, \secretize{CCC D}, \secretize{EEE F}, et al. GHIJKLMN[J]. OPQ, 2023, 111: 222-232.
    
    {\hfill(CCF-D 期刊,SCI 期刊,第一作者,已发表)}
\end{enumerate}

\subsection*{攻读硕士学位期间专利、软著申请情况}
\begin{enumerate}
  \item AAABBB专利
  
  {\hfill(第一发明人,已受理)}
\end{enumerate}


\subsection*{攻读硕士学位期间参加科研项目情况}

\begin{enumerate}
  \item AAA-BBB系统(已交付)
\end{enumerate}


\end{document}
