\documentclass[
  lang=cn,
  degree=master,
  % zhuanshuo,  % 专硕打开
  % blindtrail, % 盲审打开
  % signname,   % 承诺书签字打开
  % signdate,   % 承诺书签日期打开
  blankleft     % 空白页没页眉页脚
  openright,twoside % 章节在奇数页,页码靠外侧
]{nuaathesis}

% Windows下可以加入以下:字体设置为宋体
\setCJKmainfont[AutoFakeBold=true]{SimSun} 
\renewcommand\songti{\CJKfamily{\CJKrmdefault}}

\raggedbottom
\graphicspath{{./fig/},{./logo/},{../logo/}}

\input{global.tex}

\begin{document}

\makecover
\makedeclare
\frontmatter
\makeabstract
% 如果需要调整目录层级数量的话,取消下一行注释,数字含义: 0=chapter, 1=section, 2=subsection
% \setcounter{tocdepth}{1}
\expandafter\nuaatableofcontents
\expandafter\nuaalistoffigurestables
\include{content/abbr}

\mainmatter

% 自由修改添加的章
\include{content/cp1}
\include{content/cp2}
\include{content/cp3}
\include{content/cp4}

\appendix
% 如果需要附录的话,在这里 include
% \include{content/ex_postscript}

\backmatter
% 如果参考文献使用 biber
\bibliographystyle{nuaabib}   % 参考文献的样式
\bibliography{bib/ref}   % 参考文献,即 bib/ref.bib 文件(纯文本)

\include{content/acknowledge}
\chapter{在学期间的研究成果及发表的学术论文}

% 不要在意为什么没有 section,因为只是套用一下格式

\subsection*{攻读硕士学位期间发表(录用)论文情况}

\begin{enumerate}
    \item \textbf{\secretize{AAA B}}, \secretize{CCC D}, \secretize{EEE F}, et al. GHIJKLMN[J]. OPQ, 2023, 111: 222-232.
    
    {\hfill(CCF-D 期刊,SCI 期刊,第一作者,已发表)}
\end{enumerate}

\subsection*{攻读硕士学位期间专利、软著申请情况}
\begin{enumerate}
  \item AAABBB专利
  
  {\hfill(第一发明人,已受理)}
\end{enumerate}


\subsection*{攻读硕士学位期间参加科研项目情况}

\begin{enumerate}
  \item AAA-BBB系统(已交付)
\end{enumerate}


\end{document}
